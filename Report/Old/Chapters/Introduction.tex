
\subsection{Introduction}
There are very few programming languages which can claims themselves as 'Simple' and yet 'Powerful'. One of them is Python.
\\
Python is a very powerful programming language and easy to learn too. It is an interpreted, interactive which uses Object-oriented programming in a simple but effective manner. It includes modules, exceptions, dynamic typing, very high level data structures, dynamic data types, and classes.  It is portable too. Python's simple and stylish syntax,it's dynamic type checking along with it's interpreted nature make it an perfect language for scripting, which can be rapidly developed in various fields on various platforms like Unix variants, Windows, OS/2, Mac etc.
\subsection{History}
Python is a not very old language,released by its designer, Guido Van Rossum, in February 1991. He was then  working for CWI also known as Stichting Mathematisch Centrum. If you are wondering how the language got its name, well it's not from those dangerous reptiles. Python actually got it's name from a BBC comedy series from the seventies "Monty Python's Flying Circus". Rossum needed a name that was short, unique, and slightly mysterious. Since he was a fan of the show he thought this name was great.
Python is actually a successor of an interpreted language called ABC. Rossum, at that time,  was working in ABC and wanted to correct some of ABC's problems and keep some of its features. He was working on the Amoeba distributed operating system group and was looking for a scripting language with a syntax like ABC but with the access to the Amoeba system calls, so he decided to create a language that was generally extensible. In 1989, during the Christmas holidays, he decided to give it a try and design a language which he later called Python.
\subsection{Features of Python}
\textit{Simple}   ::	Python language is very easy to understand. Reading a good python script almost feels like reading English language itself(although very strict English). This actually is one of the strengths of Python, which allows the programmer to concentrate on the solution of the problem rather than concentrating on the syntax of the language used.
\\
\linebreak
\textit{Easy to learn}	::	As already mentioned Python language has simple syntax which is very easy to learn or understand.
\\
\linebreak
\textit{Free and Open Source}	::	Python is an example of FLOSS i.e. Free/Libré and Open Source Soft-
ware. In simple terms, we  can say that any one can distribute it's free copies, without the need dof any license. Anyone can see or even modify it's source code or use any of it's module for his/her application for free. It is based on the concept of 'knowledge shared by a group of people'. It thus constantly improves the Python Language and is responsible for release of it's new versions.
\\
\linebreak
\textit{High-level Language}	::	Python is a High-level Language. That means while writing a python program , we never had to worry about it's low level details such as ,memory management etc. 
\\
\linebreak
\textit{Portable}	::	As stated earlier, Python is free and open-source language, it has been ported to many platforms. One can use Python on Windows, Linux, Macintosh, OS/390, OS/2, Solaris and many others. 
\\
\linebreak
\textit{Interpreted}	::	Unlike languages like C or C++, in which the program is first converted from source language to binary code by a compiler, Python does not need any compilation binary code. It internally the source language in to an intermediate form called bytecode which further is translated to native language of our computer and runs it.
We will discuss this part in detail in next chapter.
\\
\linebreak
\textit{Object Oriented}	::	Python supports both procedure-oriented as well as object-oriented programming. Object Oriented Programming(OOP) uses objects which combines data-fields, methods and their functionality. when compared to languages like C++ or JAVA, Python has simple but very powerful way of doing OOP.
\\
\linebreak
\textit{Extensible}	::	We can include or use programs written in C/C++ in our Python program.
\\
\linebreak
\textit{Embeddable}	::	A user can have scripting capabilities by embedding Python programs within C/C++ programs. 
\\
\linebreak
\textit{Extensive Libraries}	::	Python has a very rich Standard library. It has various things involving regular expressions, documentation, unit testing, threading, browsers, CGI(Common Gateway Interface), GUI(Graphical User Interface), emails and many more. All these things are available where Python is installed.
Besides, it's standard library, there are many others high-level libraries such as wxPython, Python Imaging Library etc.
\\
\linebreak
\subsection{Programming}
